% Chapter 3

\chapter{Math Test Chapter} % Chapter title

\label{ch:mathtest} % For referencing the chapter elsewhere, use \autoref{ch:mathtest}

%----------------------------------------------------------------------------------------

\lipsum[13]

%----------------------------------------------------------------------------------------

\section{Some Formulas}

Due to the statistical nature of ionisation energy loss, large fluctuations can occur in the amount of energy deposited by a particle traversing an absorber element\footnote{Examples taken from Walter Schmidt's great gallery: \\ \url{http://home.vrweb.de/~was/mathfonts.html}}.  Continuous processes such as multiple scattering and energy loss play a relevant role in the longitudinal and lateral development of electromagnetic and hadronic showers, and in the case of sampling calorimeters the measured resolution can be significantly affected by such fluctuations in their active layers.  The description of ionisation fluctuations is characterised by the significance parameter $\kappa$, which is proportional to the ratio of mean energy loss to the maximum allowed energy transfer in a single collision with an atomic electron: \graffito{You might get unexpected results using math in chapter or section heads. Consider the \texttt{pdfspacing} option.}
\begin{equation}
\kappa =\frac{\xi}{E_{\mathrm{max}}} %\mathbb{ZNR}
\end{equation}
$E_{\mathrm{max}}$ is the maximum transferable energy in a single collision with an atomic electron.
\[E_{\mathrm{max}} =\frac{2 m_{\mathrm{e}} \beta^2\gamma^2 }{1 + 2\gamma m_{\mathrm{e}}/m_{\mathrm{x}} + \left ( m_{\mathrm{e}} /m_{\mathrm{x}}\right)^2}\ ,\]
where $\gamma = E/m_{\mathrm{x}}$, $E$ is energy and $m_{\mathrm{x}}$ the mass of the incident particle, $\beta^2 = 1 - 1/\gamma^2$ and $m_{\mathrm{e}}$ is the electron mass. $\xi$ comes from the Rutherford scattering cross section and is defined as:
\begin{eqnarray*} \xi  = \frac{2\pi z^2 e^4 N_{\mathrm{Av}} Z \rho
\delta x}{m_{\mathrm{e}} \beta^2 c^2 A} =  153.4 \frac{z^2}{\beta^2}
\frac{Z}{A}
\rho \delta x \quad\mathrm{keV},
\end{eqnarray*}
where

\begin{tabular}{ll}
$z$ & charge of the incident particle \\
$N_{\mathrm{Av}}$ & Avogadro's number \\
$Z$ & atomic number of the material \\
$A$ & atomic weight of the material \\
$\rho$ & density \\
$ \delta x$ & thickness of the material \\
\end{tabular}

$\kappa$ measures the contribution of the collisions with energy transfer close to $E_{\mathrm{max}}$.  For a given absorber, $\kappa$ tends towards large values if $\delta x$ is large and/or if $\beta$ is small.  Likewise, $\kappa$ tends towards zero if $\delta x $ is small and/or if $\beta$ approaches $1$.

The value of $\kappa$ distinguishes two regimes which occur in the description of ionisation fluctuations:

\begin{enumerate}
\item A large number of collisions involving the loss of all or most of the incident particle energy during the traversal of an absorber.

As the total energy transfer is composed of a multitude of small energy losses, we can apply the central limit theorem and describe the fluctuations by a Gaussian distribution. This case is applicable to non-relativistic particles and is described by the inequality $\kappa > 10 $ (\ie, when the mean energy loss in the absorber is greater than the maximum energy transfer in a single collision).

\item Particles traversing thin counters and incident electrons under any conditions.

The relevant inequalities and distributions are $ 0.01 < \kappa < 10 $, Vavilov distribution, and $\kappa < 0.01 $, Landau distribution.
\end{enumerate}

%----------------------------------------------------------------------------------------

\section{Various Mathematical Examples}

If $n > 2$, the identity \[t[u_1,\dots,u_n] = t\bigl[t[u_1,\dots,u_{n_1}], t[u_2,\dots,u_n] \bigr]\] defines $t[u_1,\dots,u_n]$ recursively, and it can be shown that the alternative definition \[t[u_1,\dots,u_n] = t\bigl[t[u_1,u_2],\dots,t[u_{n-1},u_n]\bigr]\] gives the same result.