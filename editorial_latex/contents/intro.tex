

\chapter{Introduction}
	\section{Généralités}

		La reconnaissance des formes dans les vidéos est un problème important en vision artificielle et en traitement d'images. Cette tâche est très utile vue l'accroissement du nombre de vidéos générées par les médias numériques (ex., internet, la télévision, les vidéos personnelles, la surveillance vidéo). La reconnaissance automatique des objets en vidéos peut ainsi renforcer la sécurité, faciliter la gestion des vidéos ainsi que permettre de nouvelles applications en interaction personne/machine.
		
		Par ailleurs, les images numériques et la vidéo sont devenues indispensables pour divers domaines d'application, tels que la détection d'intrusions pour la sécurité, la surveillance du trafic routier, la médecine pour l'imagerie médicale, ou encore lors des événements sportifs (ex., renforcement de l'arbitrage, création automatique de résumés).
		Des contraintes d'exploitation découlent des observations citées ci-dessus, parmi lesquelles nous citerons celles qui sont liées à la reconnaissance des objets en mouvement dans les vidéos. Par exemple, de nos jours, un très grand nombre de caméras est déployé exclusivement pour la surveillance vidéo.\cite{ahadjitse2013reconnaissance} 
		
		Souvent, le contenu de ces vidéos est interprété par des opérateurs humains qui engendrent des coûts exorbitants pour le suivi et l'analyse du contenu, sans mentionner les erreurs qui peuvent être induites par la fatigue et l'inattention humaine. 
		Un des problèmes importants abordés dans la surveillance vidéo est la reconnaissance des types d'objets en mouvement et leurs actions, afin de détecter, par exemple, des menaces potentielles (ex., vols, attentats, accidents), ou tout simplement pour des fins de statistiques (ex., compter le nombre d'individus, de voitures dans une entrée de parc).
		
		??? parler de existant ?
		
		

	\section{Contexte de notre recherche}
		Au cours de la dernière décennie, la taille des données a augmenté plus rapidement que la vitesse des processeurs. 
		Dans ce contexte, faire un traitement de reconnaissance des formes sur des vidéos en temps réel, les ensembles de données d'entraînement pour les problèmes de détection d'objets sont généralement très volumineux et les capacités des méthodes d'apprentissage automatique statistique sont limitées par le temps de calcul plutôt que par la taille de l'échantillon.\cite{bottou2010large} 
	
	\section{Problématique}
		Pour le système de vision humain, la reconnaissance des objets est une tâche simple et triviale. L'humain est capable de faire la distinction, d'une part, entre des objets et l'arrière-plan d'une image et d'autre part, entre plusieurs objets présents dans une scène de vidéo.
		??? pourquoi faire un système de reconnaissance des formes ?
		??? qu'est-ce que ça résout comme problème ?
	
	\section{Objectifs}
		Ce travail vise à proposer une méthode  intelligent basé sur l'intelligence artificielle pour une reconnaissance des plaques d’immatriculation des véhicules à l'aide du classificateur de descente de gradient stochastique Ridge-Adaline (Ridge Adaline Stochastic Gradient Descent ou RASGD).
		Pour minimiser la fonction de coût du classificateur, le RASGD adopte un modèle d'optimisation sans contrainte. De plus, pour augmenter la vitesse de convergence du classificateur, le classificateur de descente de gradient stochastique Adaline, (Adaline Stochastic Gradient Descent) est intégré à Ridge Régression.\cite{deepa2021ai}