% Table of Contents - List of Tables/Figures/Listings and Acronyms

\refstepcounter{dummy}

\pdfbookmark[1]{\contentsname}{tableofcontents} % Bookmark name visible in a PDF viewer

\setcounter{tocdepth}{2} % Depth of sections to include in the table of contents - currently up to subsections

\setcounter{secnumdepth}{3} % Depth of sections to number in the text itself - currently up to subsubsections

\manualmark
\markboth{\spacedlowsmallcaps{\contentsname}}{\spacedlowsmallcaps{\contentsname}}
\tableofcontents 
\automark[section]{chapter}
\renewcommand{\chaptermark}[1]{\markboth{\spacedlowsmallcaps{#1}}{\spacedlowsmallcaps{#1}}}
\renewcommand{\sectionmark}[1]{\markright{\thesection\enspace\spacedlowsmallcaps{#1}}}

\clearpage

\begingroup 
\let\clearpage\relax
\let\cleardoublepage\relax
\let\cleardoublepage\relax

%----------------------------------------------------------------------------------------
%	List of Figures
%----------------------------------------------------------------------------------------

\refstepcounter{dummy}
%\addcontentsline{toc}{chapter}{\listfigurename} % Uncomment if you would like the list of figures to appear in the table of contents
\pdfbookmark[1]{\listfigurename}{lof} % Bookmark name visible in a PDF viewer


%#
\listoffigures


\vspace{8ex}
\newpage

%----------------------------------------------------------------------------------------
%	List of Tables
%----------------------------------------------------------------------------------------

%\refstepcounter{dummy}
%\addcontentsline{toc}{chapter}{\listtablename} % Uncomment if you would like the list of tables to appear in the table of contents
%\pdfbookmark[1]{\listtablename}{lot} % Bookmark name visible in a PDF viewer

%\listoftables
        
%\vspace{8ex}
%\newpage
    
%----------------------------------------------------------------------------------------
%	List of Listings
%---------------------------------------------------------------------------------------- 

%\refstepcounter{dummy}
%\addcontentsline{toc}{chapter}{\lstlistlistingname} % Uncomment if you would like the list of listings to appear in the table of contents
%\pdfbookmark[1]{\lstlistlistingname}{lol} % Bookmark name visible in a PDF viewer

%\lstlistoflistings 

%\vspace{8ex}
%\newpage
       
%----------------------------------------------------------------------------------------
%	Acronyms
%----------------------------------------------------------------------------------------

\refstepcounter{dummy}
%\addcontentsline{toc}{chapter}{Acronyms} % Uncomment if you would like the acronyms to appear in the table of contents
\pdfbookmark[1]{Acronyms}{acronyms} % Bookmark name visible in a PDF viewer

\markboth{\spacedlowsmallcaps{Acronyms}}{\spacedlowsmallcaps{Acronyms}}

\chapter*{Liste des acronymes}

\begin{acronym}[UML]
	\acro{IA}{Intelligence Artificielle}
	\acro{AI}{Artificial Intelligence}
	\acro{SA}{Supervised Learning}
	\acro{AI}{Artificial Intelligence}
	
	\acro{ML}{Machine Learning}
	\acro{CV}{Computer Vision}
	
	\acro{OCR}{Optical Character Recognition}
	\acro{ANPR}{Automatic Number-Plate Recognition}
	\acro{ALPR}{Automatic License Plate Recognition}
	
	
	\acro{GD}{Gradient Descent}
	\acro{SGD}{Stochastic Gradient Descent}
	\acro{ADALINE}{ADAptative LInear NEuron }
	
	\acro{CNN}{Convolutional Neural Network}
	\acro{ANN}{Artificial Neural Network}
	\acro{VGG}{Visual Geometry Group}
	\acro{ILSVRC}{ImageNet Large Scale Visual Recognition Challenge}
	
	\acro{NAG}{Nesterov Accelerated Gradient}
	
	\acro{ReLu}{Rectified Linear Unit}
	\acro{MSE}{Mean Squared Error}
	\acro{MLP}{MultiLayer perceptron}
	
	\acro{SLNN}{Single-Layer Neural Network}
	
	\acro{DL}{Développment Limité}
	
	%\acro{NAG}{}
	%\acro{NAG}{}
	
	
	\acro{API}{Application Programming Interface}
	
	
	\acro{UML}{Unified Modeling Language}
\end{acronym} 
                   
\endgroup


%----------------------------------------------------------------------------------------
%	Symbol
%----------------------------------------------------------------------------------------
\refstepcounter{dummy}
%\addcontentsline{toc}{chapter}{Acronyms} % Uncomment if you would like the acronyms to appear in the table of contents
\pdfbookmark[1]{Notions}{notions} % Bookmark name visible in a PDF viewer

\markboth{\spacedlowsmallcaps{Notions}}{\spacedlowsmallcaps{Notions}}

\chapter*{Notions}
%\resizebox{\textwidth}{!}



\begin{tabular}{l p{0.85\linewidth}}
	$\mathbb{N} $ & Ensemble des entiers naturels\\
	%$\mathbb{R} $ &  \\
	$\mathbb{R}^n $ & Ensemble des réels ou  Espace euclidien de dimension $n$ \\
	$\mathbb{B}^n = \{0,1\}^n $ & Espace booléen de dimension $n$\\
	$\mathcal{O}(\cdot) $ ou $ {\Omega}(\cdot) $ & L'ordre de grandeur maximal de complexité d'un algorithme \\
	%$\mathcal{O}(\cdot) $ & Le grand O de la notation de Landeau \\
	
	
	$x = \begin{pmatrix}
		x_1 \\ \vdots \\ x_n 
	\end{pmatrix} $ & Un vecteur\\ 
	$x =  (x_1, \vdots,  x_n)^T$ & Un vecteur \\ 
	$x^T =  (x_1, \vdots,  x_n)^T$ & Un vecteur transposé \\ 
	$ \langle xy\rangle = x^Ty$  & Le produit vectoriel \\ 
	$\parallel x \parallel $ & La norme du vecteur\\
	
	 
	$M^{-1}$ & La matrice inverse d'une matrice $M$\\
	$M^{T}$ & La matrice transposée \\
	
	$\frac{\partial}{\partial x}f(x,y) $ & La dérivée partielle par rapport à x de al fonction $f$ des deux variable $x$ et $y$ \\ 
	
	$\nabla_A J(A,B) $ & Le vecteur dérivé par rapport au vecteur $A$ de la fonctionnelle $J$ des deux vecteurs $A$ et $B$ \\ 
	
	$ $ & \\
	
	 $ $ & \textbf{  \ \ \ \ \ \ \ \ \textsc{Les éléments en apprentissage}} \\
	
	$ \mathcal{S} $ & L'échantillon d'apprentissage (un ensemble ou une suite d'exemple)  \\ 
	$ \hat{y}$ & La valeurs prédite après l'entrainement d'un modèle d'apprentissage automatique \\ 
	$ \hat{y}_i \in  \mathcal{Y} $ & La prédiction, ou sortie désirée, d'un exemple \\
	$ \mathcal{H} $ &  Espace des hypothèses d'apprentissages \\
	$ h \in \mathcal{H} $ & Une hypothèses produite par un algorithme d'apprentissage \\
	$ y = h(x) \in \mathcal{Y} $ & La prédiction faite par l'hypothèse $h$ sur la description $s$ d'un exemple \\
	$ \ell(f(x),h(x)) $ & La fonction perte ou fonction coût entre la fonction  cible et une hypothèse sur $x$ d'un exemple \\
	
	$ \mathcal{C} $ & L'ensemble des classes \\
	$ C $ & Le nombre de classes \\
	$ \mathit{c_i} \in \mathcal{C}  $ & Une sous classe de $\mathcal{C}$ \\
	
	
\end{tabular}
