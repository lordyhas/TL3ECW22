% Colophon (a brief description of publication or production notes relevant to the edition)

\pagestyle{empty}

\hfill

\vfill

\pdfbookmark[0]{Colophon}{colophon}

\section*{Colophon}
Cette étude a été très enrichissant pour moi, car il m'a permis de découvrir le domaine du Machine Learning, apprentissage supervisé, ses acteurs. Elle m'a permis de participer concrètement à ses enjeux au travers mes missions en apprentissage supervisé et la vision par ordinateur. Je préfère ainsi m'orienter vers un domaine lié à ma mission en Calcul Scientifique et le Data Science.



%This document was typeset using the typographical look-and-feel \texttt{classicthesis} developed by Andr\'e Miede. The style was inspired by Robert Bringhurst's seminal book on typography ``\emph{The Elements of Typographic Style}''. \texttt{classicthesis} is available for both \LaTeX\ and \mLyX: 

\begin{center}
\url{https://www.katangamining.com}
\end{center}
{Ce mémoire pour le travail de fin de cycle \textsf{$2021-2021$} qui traite la thématique du \texttt{Machine Learning \& Computer Vision} a été rédigé par {TSHELEKA KAJILA Hassan}, étudiant de l'\textsc{Université Nouveaux Horizons}, conformément aux exigences du diplôme de Licencié en Sciences informatique, département : \emph{Calcul Scientifique}.}


\begin{center}
\url{https://www.unhorizons.org}
\end{center}
 
\bigskip

\noindent\finalVersionString