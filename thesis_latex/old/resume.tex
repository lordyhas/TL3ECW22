\addcontentsline{toc}{chapter}{Résumé}
\chapter*{Résumé} 
	
	Les récents progrès technologiques cationiques dans le domaine de l'information et de la communication ont introduit des moyens intelligents de gérer divers aspects de la vie.
	
	Les appareils et applications intelligents font désormais partie intégrante de notre vie quotidienne.
	
	Au cours de la dernière décennie, la taille des données a augmenté plus rapidement que la vitesse des processeurs. 
	
	%
	Dans ce contexte, faire un traitement de reconnaissance des formes sur des vidéos en temps réel, les capacités des méthodes d'apprentissage automatique statistique sont limitées par le temps de calcul plutôt que par la taille de l'échantillon. Une analyse plus précise révèle des compromis qualitativement différents pour le cas des problèmes d'apprentissage à petite et à grande échelle. Le cas à grande échelle implique la complexité de calcul de l'algorithme d'optimisation sous-jacent de manière non triviale. Des algorithmes d'optimisation improbables tels que la descente de gradient stochastique montre des performances étonnantes pour les problèmes à grande échelle. En particulier, le gradient stochastique du second ordre et le gradient stochastique moyenné sont asymptotiquement efficaces après un seul passage sur l'ensemble d'apprentissage.
	
	%
	Ce travail vise à proposer un système intelligent basé sur l'intelligence artificielle (IA) pour une reconnaissance des plaques d’immatriculation des véhicules à l'aide du classificateur de descente de gradient stochastique Ridge Adaline (Ridge Stochastic Gradient Descent Adaline aka RASGD).
	
	%
	Pour minimiser la fonction de coût du classificateur, le RASGD adopte un modèle d'optimisation sans contrainte. De plus, pour augmenter la vitesse de convergence du classificateur, le classificateur de descente de gradient stochastique Adaline, (Adaline Stochastic Gradient Descent) est intégré à Ridge Régression.
	
	Enfin, pour valider l'efficacité de système intelligent, les résultats du schéma proposé ont été comparés avec des algorithmes d'apprentissage automatique de pointe tels que Support Vector Machine (SVM) et méthodes de régression logistique. Le système intelligent RASGD atteint une précision de 92%, ce qui est meilleur que les autres classificateurs sélectionnés.
	
	
	
	
	\vspace{2 cm}

	\begin{singlespace}
		\textbf{Mots clés~:} Suppervised Learning, Computer Vision, Stocastic Gradiant Descent, Objects recognition. 
	\end{singlespace}